% ================================
% Conceptual Framework
% ================================

\section*{Conceptual Framework}
\addcontentsline{toc}{section}{Conceptual Framework}

\subsection*{Positioning the Concept of the Ideational Bricoleur}

The central conceptual contribution of this thesis is the specification of the 
\textit{ideational bricoleur}. This figure links discursive frames 
(diagnostic, prognostic, motivational) with institutional recombination 
(layering, patching, transposition) and observable uptake 
(textual reuse, funding allocation, institutional incorporation, representation). 
Discursive Institutionalism provides the macro logic of ideas in, through, and over 
institutions, while Institutional Bricolage specifies the micro practices of 
recombination. The analytical claim is that bricolage modes and their uptake 
systematically vary by institutional density and fragmentation.

\paragraph{Differentiation from adjacent concepts.}
The \textit{ideational bricoleur} is distinct from other well-established concepts.  
Unlike \textbf{policy entrepreneurs} \parencite{Kingdon1995}, whose defining feature is 
strategic coupling of problem, policy, and political streams, bricoleurs are 
identified by the substantive recombination of heterogeneous elements 
(e.g., legal instruments + scientific evidence + customary norms).  
Unlike \textbf{discourse coalition brokers} \parencite{Hajer1995}, who stabilize shared 
storylines across networks, bricoleurs operate at the junction of discourse and 
institutional practice, assembling fragments into workable arrangements.  

The added explanatory value lies in clarifying a mechanism that neither policy 
entrepreneurship nor discourse coalition theory makes visible: the translation of 
discursive frames into institutional instruments with empirically observable uptake.

\paragraph{Illustrative example.}
For instance, in Chile, proposals that combine international certification schemes 
with indigenous co-management councils exemplify \emph{ideational bricolage}. 
Such recombination differs from policy entrepreneurship (focused on agenda windows) 
and from discourse coalitions (focused on shared storylines), because it assembles 
heterogeneous instruments into a concrete governance design.

\paragraph{Analytical bridge.}
The bricoleur concept serves as a bridge, not a replacement. Policy entrepreneurship, 
epistemic communities, and discourse coalitions are employed only as auxiliary lenses 
to test boundary conditions; they are not part of the causal model. The explanatory 
focus rests squarely on bricolage as a mechanism of discursive translation into 
governance uptake. This delimitation prevents conceptual dilution while signaling 
awareness of adjacent frameworks.

\paragraph{Causal expectations.}
The framework predicts:  
\begin{itemize}
    \item In dense systems (Germany), bricolage will manifest primarily as layering 
    and patching, producing broader but incremental uptake.  
    \item In fragmented systems (Chile), bricolage will manifest more often as 
    transposition or novel assemblages, producing selective but more variable uptake.  
    \item Across cases, state actors will generally achieve higher uptake due to venue 
    access, but this gap will narrow in fragmented systems where NGOs and academics 
    substitute governance capacity.  
\end{itemize}

In sum, the \textit{ideational bricoleur} captures the micro-level practices through 
which discursive innovations are assembled into institutional proposals and tested in 
policy arenas. This provides a clear mechanism linking discourse, bricolage, and 
uptake, directly operationalized through the Uptake Index.


\paragraph{Contribution to broader debates.}
Beyond forest governance, this framework contributes to wider discussions in 
climate governance and North–South epistemologies by showing how ideas travel, 
hybridize, and gain uptake across highly divergent institutional contexts.