% ================================
% Propositions
% ================================

\section*{Propositions}
\addcontentsline{toc}{section}{Propositions}

The propositions extend prior literature but introduce a novel dimension: for the 
first time, uptake is measured systematically with a replicable \textbf{Uptake Index}. 
This enables testing not only whether bricolage differs across contexts, but also 
how discursive recombinations translate into measurable governance outcomes.

\begin{itemize}
    \item \textbf{P1: Institutional density conditions bricolage modes.}  
    In contexts of high institutional density (Germany), bricolage will manifest 
    primarily as layering or patching. In fragmented contexts (Chile), bricolage 
    will manifest more often as transposition or novel assemblages.  
    \textit{Novelty:} Unlike prior descriptive claims, this proposition will be tested 
    by comparing Uptake Index clustering patterns across cases.

    \item \textbf{P2: Actor type shapes discursive recombination.}  
    NGOs combine justice/rights frames with ecological claims, while industry 
    combines efficiency/productivity with market instruments.  
    \textit{Novelty:} The Uptake Index disaggregates how these recombinations travel 
    through different uptake channels (representation vs. funding vs. textual reuse).

    \textit{Illustrative example:} In the Chilean context, proposals linking 
    forest certification schemes with indigenous community councils could 
    hypothetically exemplify transposition bricolage. This remains to be 
    tested empirically in the coding stage, but illustrates how the concept 
    differs from policy entrepreneurship (focused on agenda timing) or 
    discourse coalitions (focused on storyline stabilization).

    \item \textbf{P3: Venue strategy influences uptake pathways.}  
    Actors targeting low-veto venues (advisory councils, certification) will achieve 
    higher uptake than those targeting high-veto venues (parliamentary arenas).  
    \textit{Novelty:} This is the first attempt to quantify venue effects using a 
    standardized uptake metric.
\end{itemize}

\subsection*{Comparative Strategy}

The research design follows a structured-focused comparison:  
\textit{structured}, because each actor type is examined with the same set of guiding questions;  
\textit{focused}, because the questions derive directly from the conceptual framework.  
This ensures comparability while preserving contextual specificity.

For each actor type (state, NGOs, business, science) in Germany and Chile, the analysis asks:

\begin{enumerate}
    \item Which discursive frames are mobilized? 
    \item Which institutional instruments are invoked? 
    \item How are these elements recombined (layering, patching, transposition, novel assemblage)? 
    \item Which governance venues are targeted? 
    \item What observable uptake can be identified? 
\end{enumerate}

This design prevents the analysis from devolving into parallel narratives. 
Instead, it enables systematic comparison of bricolage practices across institutional contexts, 
highlighting not whether Chile deviates from Germany, but how different governance logics 
produce distinct forms of bricolage and uptake.

\paragraph{Novelty.} 
These propositions are not mere extensions of prior literature: they operationalize 
\emph{uptake} for the first time with a replicable index. No existing study has systematically 
tested bricolage across North–South contexts with a standardized Uptake Index (so far). 
This makes the contribution both theoretical and methodological.