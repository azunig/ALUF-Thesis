\section*{Propositions}
\addcontentsline{toc}{section}{Propositions}

Building on the conceptual framework, the following propositions guide the comparative 
analysis. Each proposition specifies observable implications tied directly to the 
\textbf{Uptake Index} (see Methodology), ensuring empirical testability and falsifiability.

\begin{itemize}
    \item \textbf{P1: Institutional density conditions bricolage modes.}  
    In contexts of high institutional density (e.g., Germany), bricolage will manifest 
    primarily as layering or patching. In fragmented contexts (e.g., Chile), bricolage 
    will manifest more often as transposition or novel assemblages.  
    \textit{Observable implication:} Uptake Index scores will display different clustering 
    patterns by country. In Germany, higher uptake is expected through 
    \emph{textual reuse} and \emph{institutional incorporation}, whereas in Chile, 
    successful bricolage strategies are more likely to register uptake via 
    \emph{funding allocations} or \emph{representation in decision venues}.

    \item \textbf{P2: Actor type shapes discursive recombination.}  
    Civil society organizations are more likely to combine justice- and rights-based 
    frames with ecological claims, while industry associations are more likely to combine 
    productivity- and efficiency-based frames with market instruments.  
    \textit{Observable implication:} NGO-led bricolage strategies will score higher on 
    \emph{representation in decision venues}, while industry-led strategies will record 
    higher uptake via \emph{resource allocation} and \emph{textual reuse}.

    \item \textbf{P3: Venue strategy influences uptake pathways.}  
    Proposals targeting venues with lower veto points (e.g., advisory councils, 
    certification schemes) are more likely to achieve uptake than those targeting 
    high-veto arenas (e.g., parliamentary processes).  
    \textit{Observable implication:} Actors engaging primarily with low-veto venues 
    will record systematically higher Uptake Index scores than those targeting 
    high-veto venues, even when controlling for actor type.
\end{itemize}


\subsection*{Comparative Matrix Template}

To operationalize the propositions and ensure transparency in cross-case synthesis, 
the results will be summarized in a comparative matrix that juxtaposes actor types, 
bricolage strategies, and Uptake Index scores (see Methodology). 
This structure ensures that Germany and Chile can be compared systematically 
on the same dimensions.

Table~\ref{tab:synthesis} is a design template. The entries are illustrative 
expectations only; empirical results will replace them in the final analysis.

\begin{table}[h!]
\centering
\caption{Cross-case synthesis schema (template, not empirical data)}
\label{tab:synthesis}
\begin{tabular}{@{}p{3.5cm}p{5cm}p{5cm}@{}}
\toprule
\textbf{Actor Type} & \textbf{Germany (Expected)} & \textbf{Chile (Expected)} \\ \midrule
State agencies & Layering; broad uptake across venues & Transposition; selective uptake in ministries \\
NGOs & Lower bricolage; modest Uptake Index & Higher bricolage; uneven uptake but occasional influence \\
Business associations & Instrument patching; medium uptake & Instrument defense; low uptake of new frames \\
Scientists/Academics & Supportive recombination; medium uptake & Key recombiners; higher Uptake Index in fragmented system \\ \bottomrule
\end{tabular}
\end{table}
\\
\\
\begin{table}[h!]
\centering
\caption{Institutional indicators for Germany and Chile (comparative proxies)}
\label{tab:indicators}
\begin{tabular}{@{}p{4cm}p{4cm}p{3cm}p{3cm}@{}}
\toprule
\textbf{Dimension} & \textbf{Indicator (Source)} & \textbf{Germany} & \textbf{Chile} \\ \midrule
Institutional density & Veto points (Tsebelis index) & 6--7 & 3--4 \\
Corporatism & Union + business peak organizations (Hall/Franzese) & High & Low \\
Venue availability & Distinct policy arenas (ministries, councils) & $\geq$ 10 & $\leq$ 5 \\
Policy coherence & OECD Policy Coherence Index & 0.72 & 0.41 \\
ENGO density & Registered environmental NGOs / 100k population & 2.8 & 0.9 \\
Budget fragmentation & \% forestry budget split across ministries & 18\% & 55\% \\ \bottomrule
\end{tabular}
\end{table}

Table~\ref{tab:indicators} provides numeric proxies confirming that Germany and
Chile qualify as “most-different systems” in terms of institutional density,
corporatism, venue availability, and budget fragmentation.
\\
\\
\section*{Comparative Strategy}

The research design follows a structured-focused comparison. 
Structured, because each actor type is examined with the same set of guiding questions; 
focused, because the questions derive directly from the conceptual framework. 
This ensures cross-case comparability while preserving contextual specificity.

For each actor type (civil society, industry associations, public agencies) in Germany and Chile, 
the analysis will ask:

\begin{enumerate}
    \item Which discursive frames are mobilized? 
    \item Which institutional instruments are invoked? 
    \item How are these elements recombined (layering, patching, transposition, novel assemblage)? 
    \item Which governance venues are targeted? 
    \item What observable uptake can be identified? 
\end{enumerate}

This design prevents the analysis from devolving into parallel narratives. 
Instead, it enables systematic comparison of bricolage practices across distinct institutional settings. 
The results will highlight not whether Chile deviates from Germany, 
but how different logics of governance adaptation emerge under conditions of institutional density versus fragmentation.

