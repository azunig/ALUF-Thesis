\section*{Literature Review}
\addcontentsline{toc}{section}{Literature Review}

\subsection*{Concepts and Theoretical Approaches}

Understanding the politics of climate adaptation requires a framework that connects 
structure and agency, linking the power of ideas to the actions of those who wield them. 
Several traditions provide useful entry points.

First, \textbf{Discursive Institutionalism (DI)} highlights how institutions are not only 
sets of rules but also carriers of ideas and discourses that give them meaning 
(\cite{Schmidt2008}). DI’s distinction between \textit{coordinative discourse} among 
policy elites and \textit{communicative discourse} for broader publics is essential for 
examining adaptation debates. The related concept of \textbf{Ideational Power} specifies 
the mechanisms of influence, distinguishing between persuasive power through ideas, 
power over ideas, and institutional power in ideas (\cite{CarstensenSchmidt2016}). 
Recent work has demonstrated how ideational power shapes institutional trajectories 
across governance levels (\cite{Kleinschmit2024}).

Second, the concept of \textbf{Institutional Bricolage} explains the \textit{how} of 
institutional formation. Originally coined by L\'{e}vi-Strauss and later applied to 
natural resource governance (\cite{Cleaver2001}), bricolage captures the pragmatic 
process whereby actors recombine available rules, norms, and cultural repertoires. 
This has been theorized as a strategy especially visible in contexts of crisis, where 
actors “patch” together elements from different institutional logics to produce novel 
arrangements (\cite{Carstensen2017}).

However, these two perspectives also operate at different analytical levels. 
While \textbf{Discursive Institutionalism} tends to emphasize the macro-level 
structuring power of ideas and discourses across institutions, 
\textbf{Institutional Bricolage} highlights micro-level agency and the pragmatic 
recombination of locally available norms and rules. This study argues that the 
figure of the \textit{ideational bricoleur} provides a conceptual bridge: it 
captures how actors strategically connect discursive innovation at the macro-level 
with bricolage practices at the local level, thereby making visible how ideas travel, 
hybridize, and reconfigure governance arrangements. Similar efforts to theorize the 
multi-scalar role of ideas in policy change can be found in work on ideational 
institutionalism more broadly (\cite{BelandCox2011}).

\paragraph{Theoretical integration risks.}
Analytically, Discursive Institutionalism and Institutional Bricolage form the core 
explanatory lens. Policy entrepreneurship, epistemic communities, and discourse 
coalitions are employed only as auxiliary robustness checks to probe boundary 
conditions, not as co-equal frameworks. This design mitigates the risks of dilution 
and eclecticism by clearly delimiting the focus on the \textit{ideational bricoleur} 
as the central synthesis. This study explicitly delimits DI + bricolage as core,
employing other frameworks only as robustness checks


Third, alternative frameworks provide adjacent perspectives on agency. 
\textbf{Policy entrepreneurs} emphasize the strategic coupling of problems, 
policies, and politics to open ``windows of opportunity'' 
(\cite{Kingdon1995,MintromNorman2009}). 
\textbf{Epistemic communities} focus on networks of experts who exert influence 
through shared causal beliefs and professional authority (\cite{Haas1992}). 
Finally, \textbf{Discourse coalition theory} examines how actors coalesce around 
storylines that give meaning to environmental debates (\cite{Hajer1995,ArtsBuizer2009}). 

These approaches provide valuable insights but often remain either at the abstract level 
of discourse or at the practical level of institutional strategy, without fully 
integrating the two.



\subsection*{Empirical updates in forest adaptation (2020--2024)}

Recent empirical research highlights the urgency and institutional variation of forest adaptation.  
In Germany, the \textit{Report of the Scientific Advisory Board on Forest Policy: 
Adaptation of forests and forestry to climate change} (Thünen Institute, 2022) 
lays out concrete policy instruments and institutional pathways.  
At the European level, the study \textit{Forest-based Climate Change Mitigation 
and Adaptation in Europe} (EFI, Verkerk et al., 2022) reviews recent evidence 
on disturbances, governance instruments, and policy responses.  
Similarly, \textit{Governing Europe’s Forests for Multiple Ecosystem Services: 
Opportunities, Challenges, and Policy Options} (Winkel et al., 2022) emphasizes
 how adaptation, biodiversity, and multifunctionality discourses are increasingly institutionalized.  
The \textit{Forest Europe Implementation Report 2021--2024} documents national 
differences in implementing adaptation measures across signatory countries.  


On the Chilean side, the World Bank’s report \textit{Chile’s Forests: A Pillar 
for Inclusive and Sustainable Growth} (2023) shows how adaptation discourses
are linked to social inclusion and economic policy, illustrating how global 
narratives interact with local institutional legacies. This picture is reinforced 
by national policy initiatives and scientific evidence. 
The \textit{Plan Nacional de 
Restauración de Paisajes 2021--2030} (\cite{MMA2021}) and the 
\textit{Ecosystem Restoration Country Dossier} 
(\cite{MMACONAF2024}) demonstrate 
institutional pathways for adaptation. Complementing these, recent research since 2022 on 
wildfire dynamics (\cite{VidalSilva2025}), forest hydrology (\cite{Balocchi2023}), and 
biodiversity and restoration (\cite{Bosque2022Biodiversity,RCHN2023Restoration}) highlights 
the ecological urgency underpinning governance debates, showing how Chilean scholarship 
contributes not only locally but also to global adaptation governance. 
Analysis of recent fire events and atmospheric conditions (\cite{McWethy2021FiresChile}) 
and studies of institutional and resilience barriers in Araucanía (\cite{Banwell2020AraucaniaResilience}) 
further underline that adaptation is not only a scientific concern but a policy and 
governance challenge with real local constraints.

These contributions complement the conceptual frameworks of Discursive Institutionalism, 
bricolage, policy learning, and resilience thinking by providing concrete evidence for 
comparative analysis.

In addition, recent strands of scholarship have stressed the importance of 
\textbf{policy learning} and \textbf{resilience thinking} in explaining how 
governance systems adapt to climate change. Policy learning approaches highlight 
how actors update their preferences and strategies through processes of social 
interaction and institutional feedback, often across multiple levels of governance 
(\cite{DuitGalaz2020,JordanHuitema2019}). Similarly, resilience thinking has 
provided a conceptual vocabulary to understand how institutions evolve under 
conditions of uncertainty and disturbance, emphasizing adaptive capacity, 
transformability, and path-dependence. These perspectives complement discursive 
and bricolage approaches by clarifying the cognitive and institutional mechanisms 
through which ideas stabilize or shift. They further reinforce the need for an 
integrative framework---such as the figure of the \textit{ideational bricoleur}---that 
captures how discursive innovation, pragmatic recombination, and adaptive learning 
interact in shaping forest adaptation governance.

\subsection*{Forest Governance and Adaptation Studies}

Scholarship on forest governance has expanded considerably over the past three decades. 
In Europe, analyses have focused on global forest regimes and the rise of 
non-state authority such as certification systems (\cite{Cashore2004}), 
the institutionalization of discourses in European forest policy 
(\cite{ArtsBuizer2009}), and the conflicts between multifunctional forestry 
and conservation in Germany (\cite{Winkel2011}). Comparative studies have 
documented convergence and divergence across governance systems 
(\cite{HowlettRayner2006}), while multi-level perspectives have shown how 
EU, national, and local arenas interact to shape outcomes (\cite{Arts2006,Pulzl2013}). 

In Latin America, research has emphasized fragmented authority, contested tenure, 
and the dominance of market-oriented forestry models. Studies have traced how 
such institutions perpetuate inequality and marginalize indigenous knowledge 
(\cite{LarsonRibot2007,AnderssonPacheco2006,Manuschevich2016}). 
Critical Latin American perspectives underline the embeddedness of governance 
in neoliberal reforms and extractivist paradigms (\cite{Gudynas2011,Escobar1996}), 
which continue to structure the space for adaptation. These contributions highlight 
that forest governance cannot be disentangled from broader political economies 
and cultural narratives.

Climate change adaptation has entered this field only more recently. In Germany, 
adaptation discourses have largely been absorbed into long-standing forestry versus 
conservation struggles, reinforcing discursive path dependency (\cite{Winkel2011}). 
In Chile, adaptation has been mediated through plantation-centered institutions 
and narratives privileging growth, often clashing with indigenous and peasant 
worldviews (\cite{Manuschevich2016}). This reinforces the need for comparative 
research that does not merely juxtapose national cases but interrogates the ways 
in which actors use ideas and institutions to broker change.


\subsection*{Identified Gap and Justification}

While substantial insights exist, comparative research that explicitly theorizes 
the role of actors as brokers of both discourses and institutions remains scarce. 
Recent debates on multilevel forest governance (\cite{KleinschmitChiassonPulzl2023,Kleinschmit2024}) 
and new Chilean evidence on hydrology, wildfire, and biodiversity 
(\cite{Oyarzun2022Bosque,Gonzalez2022Bosque,Bosque2022Biodiversity,RCHN2023Restoration}) 
highlight the urgency of integrating adaptation across scales. 

\textbf{Yet, despite recent advances (Kleinschmit 2023; Kleinschmit et al. 2024; Oyarzún 2022; González 2022), 
no study has systematically examined the bricoleur as a comparative mechanism in North–South adaptation governance.}



This thesis introduces the figure of the \textbf{ideational bricoleur} as a 
conceptual synthesis of Discursive Institutionalism and Institutional Bricolage. 
By tracing how actors strategically recombine discourses and institutional tools 
across multiple governance levels, the concept makes visible the mechanisms by 
which ideas travel, hybridize, and occasionally transform entrenched governance 
systems. In doing so, it responds to recent scholarly debates on multilevel forest 
adaptation governance \cite{KleinschmitChiassonPulzl2023,Kleinschmit2024}.

This framework can also be extended to other domains of resource governance, 
such as water or energy, where actors similarly navigate fragmented discourses 
and institutional arrangements.



\begin{table}[h!]
\centering
\tiny
\caption{Selected literature on forest adaptation governance, bricolage, and discourse}
\label{tab:litreview}
\begin{tabular}{@{}p{2cm}p{2cm}p{3cm}p{4cm}p{4.5cm}@{}}
\toprule
\textbf{Author, Year} & \textbf{Concept} & \textbf{Context} & \textbf{Method} & \textbf{Key Findings} \\ \midrule
Winkel (2011) & Discursive institutionalism & DE (forestry policy) & Frame analysis of policy documents & Shows how discursive struggles shape forest governance; informs coding of frames. \\
Manuschevich (2016) & Institutional legacies / neoliberalism & CL (forests, water) & Policy/legal analysis & Demonstrates how neoliberal reforms created fragmented institutions; basis for case contrast. \\
Pahl-Wostl (2015) & Institutional bricolage & Comparative water governance & Comparative case studies & Introduces bricolage concept; provides typology for layering/patching/transposition. \\
Mukhtarov (2014) & Policy translation & Global/Europe & Discourse analysis & Highlights how concepts travel across contexts; relevant for Uptake Index. \\
Hall (1993) & Policy paradigms & OECD countries & Historical analysis & Shows paradigm shifts via layering; informs H1 expectations. \\
Stone (2017) & Epistemic communities & Global & Case studies & Explains role of experts; complements actor-type hypothesis (H3). \\
Hajer (1995) & Discourse coalitions & Environmental politics & Discourse analysis & Provides tools to trace coalition-building; strengthens Stage 1 actor coding. \\
\bottomrule
\end{tabular}
\end{table}


This review shows that bricolage has been applied mainly to water governance 
\parencite{PahlWostl2015} but rarely to forests. In Germany, discursive struggles 
are documented \cite{Winkel2011}, while in Chile neoliberal fragmentation is 
emphasized \cite{Manuschevich2016}. No comparative study has measured uptake 
across North–South contexts. This gap motivates the design summarized in 
Table~\ref{tab:litreview}.

