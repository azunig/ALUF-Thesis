% --- Resumen / Abstract ---
\begin{abstract}
How do actors in contrasting institutional contexts assemble discourses and
instruments for forest adaptation, and how far do their ideas travel? This
thesis compares Germany and Chile as most-different cases of forest governance.
Building on discursive institutionalism and the concept of \emph{ideational
bricolage}, it introduces an \textbf{Uptake Index} to trace how frames and
instruments circulate across policy arenas. The study analyzes a corpus of
50--60 key documents (2015--2025) through systematic coding. It is expected
that (1) Germany’s dense institutional setting fosters incremental layering of
established tools, while (2) Chile’s fragmented governance encourages
transposition and more radical recombination. These dynamics generate distinct
uptake profiles across the two cases. The contribution is twofold: theoretically,
by clarifying mechanisms of discursive translation; and methodologically, by
offering a replicable tool for assessing policy uptake.
\\
\\
\noindent\textbf{Keywords:} forest adaptation, discursive institutionalism,
ideational bricolage, Uptake Index, Germany, Chile, policy uptake
\end{abstract}