% ================================
% Introduction
% ================================

\section*{Introduction}
\addcontentsline{toc}{section}{Introduction}

Forests are at the frontline of the climate crisis: simultaneously vital carbon
sinks and highly vulnerable ecosystems. Intensifying droughts, wildfires, and
pest outbreaks are transforming forests from buffers into emission sources. In
Germany, four out of five trees were classified as diseased in 2022 due to
drought and bark beetle infestations (\cite{BMEL2023}); in Chile, a decade-long
\emph{megasequía} combined with record wildfires has placed immense stress on
both native ecosystems and plantation economies (\cite{FAO2022}). Policy
responses increasingly foreground adaptation, from Germany’s
\emph{Waldstrategie 2050} to Chile’s \emph{Plan Nacional de Restauración de
Paisajes 2021–2030} (\cite{MMA2021}). Yet little is known about
\emph{how} actors in such divergent contexts recombine discursive and
institutional elements, or how their ideas circulate across policy arenas.

Existing research emphasizes the role of institutions in shaping responses to
forest decline (\cite{Winkel2011}; Winkel \& Sotirov 2016) and the power of ideas
in legitimizing adaptation pathways (\cite{Schmidt2008}). However, less attention
has been paid to how actors strategically \emph{combine} discursive frames and
institutional instruments. Discursive institutionalism highlights the power of
ideas \emph{in, through, and over} institutions, while the notion of
\emph{ideational bricolage} underscores how actors recombine heterogeneous
resources into workable strategies. Integrating these perspectives allows for a
sharper theorization of adaptation governance.

This thesis adopts a comparative design. Germany represents a dense corporatist
context where adaptation is typically layered onto existing institutions, while
Chile exemplifies a fragmented neoliberal system where bricolage often proceeds
through transposition or novel assemblages. By tracing actors’ discursive
strategies and their observable uptake, the study explains how contrasting
governance logics shape the forms of bricolage available.

The contribution is threefold: (i) theoretical, by integrating discursive
institutionalism and bricolage into a causal framework; (ii) methodological, by
operationalizing bricolage through a transparent Uptake Index; and (iii)
empirical, by adding a Global South case to a research field dominated by
Northern perspectives. 