\section*{Introduction}
\addcontentsline{toc}{section}{Introduction}

Forests are central to the climate crisis: simultaneously vital carbon sinks and
highly vulnerable ecosystems. Intensifying droughts, fires, and pests are
transforming forests from buffers into emission sources. In Germany, four out of
five trees were classified as diseased in 2022 due to drought and bark beetle
outbreaks (\cite{BMEL2023}); in Chile, a decade-long \emph{megasequía} combined
with record wildfires has placed immense stress on both native ecosystems and
plantation economies (\cite{FAO2022}). Policy responses now foreground
adaptation, from Germany’s \emph{Waldstrategie 2050} to Chile’s \emph{Plan
Nacional de Restauración de Paisajes 2021–2030} (\cite{MMA2021}). Yet we know
little about \emph{how} actors in such divergent contexts recombine discursive
and institutional elements, or how their ideas circulate across policy arenas.
This study addresses that gap by applying discursive institutionalism and the
concept of ideational bricolage in a comparative analysis of Germany and Chile.

Existing research emphasizes the importance of institutions in shaping responses 
to forest decline (Winkel et al. 2011; Winkel \& Sotirov 2016) and the role of ideas 
in legitimizing adaptation pathways (\cite{Schmidt2008}). Yet less is known about how 
actors strategically \emph{combine} discursive frames and institutional instruments. 
Discursive Institutionalism points to the power of ideas \emph{in, through, and over} 
institutions, while the notion of \emph{ideational bricolage} highlights how actors 
recombine heterogeneous resources into workable strategies. Bringing these strands 
together allows a sharper theorization of adaptation governance: who succeeds as a 
bricoleur, under what conditions, and with what observable uptake?

This thesis addresses that gap through a comparative design. Germany represents a 
dense corporatist context where adaptation tends to be layered onto existing 
institutions, while Chile exemplifies a fragmented neoliberal context where bricolage 
often occurs through transposition or novel assemblages. By tracing actors’ discursive 
strategies and their uptake, the study explains how distinct governance logics shape 
the forms of bricolage available.


The contribution is threefold: (i) theoretical, by integrating Discursive 
Institutionalism and bricolage into a causal framework; (ii) methodological, by 
operationalizing bricolage through a transparent Uptake Index; and (iii) empirical, 
by adding a Global South case to a research field dominated by Northern perspectives.
