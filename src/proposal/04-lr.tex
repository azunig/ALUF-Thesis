% ================================
% Literature Review
% ================================

\section*{Literature Review}
\addcontentsline{toc}{section}{Literature Review}

\subsection*{Concepts and Theoretical Approaches}

Understanding the politics of forest adaptation requires connecting 
\emph{discursive institutionalism} (DI) with the pragmatics of \emph{institutional bricolage}.  

First, DI highlights how institutions are simultaneously material and ideational, 
carrying discourses that give them meaning \parencite{Schmidt2008}. The distinction 
between \textit{coordinative} discourse (among elites) and \textit{communicative} 
discourse (towards publics) clarifies adaptation debates. The related concept of 
\emph{ideational power} specifies mechanisms of influence—persuasive power through ideas, 
power over which ideas are legitimate, and institutional power embedded in rules 
\parencite{CarstensenSchmidt2016}.

Second, bricolage explains the \textit{how} of recombination. Initially developed by 
L\'{e}vi-Strauss, the concept was adapted to natural resource governance to describe 
how actors pragmatically patch together rules, norms, and practices under constraint 
\parencite{Cleaver2001,Carstensen2017}. Bricolage is particularly visible in times of 
crisis or institutional voids.

Bringing DI and bricolage together highlights the figure of the 
\textbf{ideational bricoleur}: an actor who recombines discursive frames and 
institutional instruments into workable strategies. This bridges macro-level 
discursive struggles with micro-level practices of recombination. 

Alternative frameworks—policy entrepreneurship \parencite{Kingdon1995}, epistemic 
communities \parencite{Haas1992}, discourse coalitions \parencite{Hajer1995}—inform 
specific dynamics of agency but do not explain how heterogeneous elements are 
recombined into new institutional arrangements. They remain auxiliary lenses here.

\subsection*{Empirical Updates in Forest Adaptation (2020--2024)}



In Germany, the \textit{Report of the Scientific Advisory Board on Forest Policy} 
(Thünen Institute, 2022) outlines adaptation pathways and governance challenges. 
The European Forest Institute review by Verkerk et al. (2022) synthesizes 
evidence on climate disturbances and policy instruments. Winkel et al. (2022) 
analyze multifunctionality discourses in EU policy, while the \textit{Forest Europe 
Implementation Report} (2021--2024) compares national trajectories.  

In Chile, the World Bank (2023) positions forests as pillars of sustainable growth, 
linking adaptation to social inclusion. National initiatives include the 
\emph{Plan Nacional de Restauración de Paisajes 2021--2030} \parencite{MMA2021} and 
the \emph{Ecosystem Restoration Country Dossier} \parencite{MMACONAF2024}. 
Scientific contributions emphasize wildfire dynamics \parencite{VidalSilva2025}, 
forest hydrology \parencite{Balocchi2023}, and biodiversity restoration 
\parencite{Bosque2022Biodiversity,RCHN2023Restoration}. These highlight both 
biophysical urgency and institutional fragmentation.

\subsection*{Forest Governance and Adaptation Studies}

European scholarship emphasizes the institutionalization of discourses 
\parencite{ArtsBuizer2009,Winkel2011}, certification and non-state authority 
\parencite{Cashore2004}, and multi-level policy interaction 
\parencite{Pulzl2013}. German adaptation discourses remain path-dependent, layered 
onto long-standing forestry–conservation conflicts \parencite{Winkel2011}.  

In Latin America, forest governance is marked by tenure insecurity, privatization, 
and contestation of extractive models \parencite{LarsonRibot2007,Manuschevich2016}. 
Critical perspectives stress neoliberal legacies and indigenous struggles 
\parencite{Gudynas2011,Escobar1996}. Chile illustrates how plantation-centered 
institutions shape adaptation discourses in tension with territorial rights.

Latin American scholarship further highlights community-based adaptation and 
indigenous rights (e.g., Boelens 2016; Oyarzún 2021; Banwell 2022), broadening 
the perspective beyond state-centered accounts.

\subsection*{Identified Gap and Justification}

Despite advances, no comparative research systematically examines the 
\textbf{ideational bricoleur} as a mechanism in North–South forest adaptation governance.  
Existing studies document discursive struggles in Europe \parencite{Winkel2011}, 
institutional fragmentation in Chile \parencite{Manuschevich2016}, and recent 
scholarship links adaptation to governance scales 
\parencite{Kleinschmit2024,KleinschmitChiassonPulzl2023}.  

Yet the specific micro-practices by which actors recombine discursive and 
institutional elements—and how these generate observable uptake—remain untheorized.  
This thesis addresses that gap by integrating DI and bricolage through the concept 
of the ideational bricoleur, applying it systematically in a most-different systems 
comparison of Germany and Chile.

\begin{table}[h!]
\centering
\tiny
\caption{Selected literature on forest adaptation governance, bricolage, and discourse}
\label{tab:litreview}
\begin{tabular}{@{}p{2.5cm}p{2.5cm}p{3cm}p{4cm}p{4.5cm}@{}}
\toprule
\textbf{Author, Year} & \textbf{Concept} & \textbf{Context} & \textbf{Method} & \textbf{Key Findings} \\ \midrule
Winkel (2011) & Discursive institutionalism & Germany & Frame analysis & Discursive struggles drive governance; path dependency in adaptation debates. \\
Manuschevich (2016) & Neoliberal legacies & Chile & Policy/legal analysis & Fragmented institutions constrain adaptation; plantation model dominates. \\
Pahl-Wostl (2015) & Institutional bricolage & Comparative water governance & Case studies & Typology of layering/patching/transposition; informs coding. \\
Mukhtarov (2014) & Policy translation & Europe/global & Discourse analysis & Shows how concepts travel; relevant for Uptake Index. \\
Carstensen \& Schmidt (2016) & Ideational power & Europe & Theory synthesis & Specifies mechanisms of ideational influence; supports DI link. \\
Boelens (2016) & Community water governance & Latin America & Comparative ethnography & Highlights indigenous rights, collective adaptation, bricolage practices. \\
Kleinschmit et al. (2023, 2024) & Multilevel forest governance & EU/Chile & Policy analysis & Emphasize scale interactions and adaptation governance. \\
\bottomrule
\end{tabular}
\end{table}