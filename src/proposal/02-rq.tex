\section*{Research Questions}
\begin{quote}
\textbf{RQ1 (primary).} Under what institutional conditions do actors engage in 
\emph{ideational bricolage} in forest adaptation governance?  

\textbf{RQ2 (secondary).} What specific forms of \emph{uptake} (textual reuse, funding allocation, 
institutional incorporation, representation) result from such bricolage across policy arenas?  
\end{quote}

\section*{Hypotheses}
\begin{itemize}
  \item \textbf{H1.} In dense institutional contexts (Germany), bricolage manifests as layering/patching, 
  with uptake clustered in textual reuse and institutional incorporation.  
  \item \textbf{H2.} In fragmented contexts (Chile), bricolage manifests as transposition/assemblage, 
  with uptake clustered in funding allocations and venue representation.  
  \item \textbf{H3.} Actor type conditions bricolage: NGOs combine justice-based frames with ecological claims; 
  industry combines efficiency/productivity with market instruments.  
\end{itemize}

\section*{Case Justification}
Germany and Chile are selected as a most-different systems design. Both face analogous biophysical threats 
(droughts, wildfires, forest dieback) but are embedded in sharply contrasting political–institutional contexts. 
Germany exemplifies dense corporatism; Chile, fragmented neoliberalism. This contrast maximizes analytical leverage 
and avoids Eurocentric bias.