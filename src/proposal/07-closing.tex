% ================================
% Conclusion
% ================================

\section*{Conclusion}
\addcontentsline{toc}{section}{Conclusion}

This thesis develops a comparative framework that links discursive institutionalism 
with ideational bricolage to explain forest adaptation governance. Its contribution 
is threefold: (i) a causal model specifying how institutional density and fragmentation 
shape bricolage strategies, (ii) an Uptake Index that operationalizes discursive 
influence through observable indicators, and (iii) a North–South comparative design 
that incorporates Chile alongside Germany, countering Eurocentric bias in the field.

The design is explicitly bounded: it relies on document analysis and captures uptake 
only as observable incorporation, not actor intentions. Future research using 
interviews or process-tracing would be required to validate strategic intent more 
directly. Within these limits, the framework offers a transparent and replicable 
approach that contributes both theoretically and methodologically to debates on 
adaptation governance.

Within these limits, the framework offers a transparent and replicable approach that 
contributes both theoretically and methodologically to debates on adaptation governance. 
It also lays a foundation for future PhD-level research, where interviews and process-tracing 
could build on this baseline to deepen causal inference.

