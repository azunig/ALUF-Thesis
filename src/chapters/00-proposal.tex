
\section*{1. Abstract}
\addcontentsline{toc}{section}{1. Abstract}

This thesis examines how forest adaptation to climate change is governed in Germany and Chile. Using a comparative, discourse-centered approach, it applies the theoretical lenses of discursive institutionalism, ideational power, and institutional bricolage to explore the construction and legitimation of adaptation strategies. Drawing exclusively on primary/secondary sources---including legal frameworks, published policy strategies, and institutional narratives---it analyzes how responses to climate change emerge from the interplay of discourse and institutional architectures. The objective is to advance climate governance scholarship by connecting ideational dynamics with institutional practices in two high-profile, yet structurally distinct, political contexts.

\section*{2. Introduction}
\addcontentsline{toc}{section}{2. Introduction}

Forests are at once central to climate change mitigation and acutely threatened by shifting climatic patterns. Over the last decade, Germany and Chile have experienced frequent droughts, pest outbreaks, and declines in forest health; in Germany, four out of five trees were diagnosed as diseased in 2022 due to drought and insect infestations. Both countries are known to view forest governance as vital for meeting national and international climate goals.

This thesis investigates how contrasting institutional settings shape the discourse and practice of forest adaptation in Germany and Chile. By comparing these cases, the research seeks to reveal how divergent political traditions, actor coalitions, and institutional legacies affect adaptation responses. 

\textbf{Research Aim:}

The aim is to systematically compare how institutional architectures and discursive strategies condition forest adaptation governance in Germany and Chile---identifying both differences and points of convergence.

\textbf{Research Question:}

\begin{quote}
\textit{How do ideational power and institutional bricolage---within a discursive institutionalist framework---shape forest adaptation governance in Germany and Chile, and what insights does this comparison offer on the capacity of ideas and institutional flexibility to drive climate policy?}
\end{quote}

The research question focuses on three central concepts: (1) How actors use discourse and ideas to define and legitimize adaptation; (2) how institutional bricolage enables flexible, context-sensitive responses; and (3) whether discursive institutionalism can bridge cases with fundamentally different governance models. This approach not only identifies divergences and similarities, but also clarifies whether common theoretical frameworks are applicable across diverse contexts.

\section*{3. Literature Review}
\addcontentsline{toc}{section}{3. Literature Review}

\subsection*{3.1 Concepts and Definitions}

Discursive Institutionalism (DI) posits that ideas and discourse are central drivers of institutional change. Schmidt (2008, 2010) highlights both the coordinative role among elites and communicative function towards society. Ideational Power, developed by Carstensen and Schmidt (2016, 2018), describes actors' abilities to shape agendas and beliefs through discourse---via power over, and in ideas. Institutional Bricolage, pioneered by Cleaver (2001, 2012), theorizes how governance arrangements emerge from assembling existing institutional elements in context-sensitive ways, rather than relying solely on hierarchical design.

Underlying these concepts is the insight that institutional change in forest governance is often non-linear and responsive to ecological and social pressures. Connecting them enables analysis of how policy responses can be both institutionally embedded and discursively negotiated.

\subsection*{3.2 Status Quo of Research and Forest Governance}

Research on forest adaptation governance has grown rapidly since the mid-2010s. Early studies were dominated by technical adaptation frameworks and the study of policy instruments. More recently, there is recognition of the need to foreground institutional dynamics and actor discourse. Several reviews document how climate change is shifting forest sector priorities, while meta-analyses highlight ongoing challenges, including conflicting interests between climate protection, resource extraction, and the timber industry.

In Germany, large-scale forest inventories reveal a marked decline in timber and carbon stocks since 2017, and forests have ceased to serve as a net CO$_2$ sink---now acting as a source due to climate-related dieback and intensive harvest. In Chile, scholarship emphasizes institutional fragmentation between public agencies (CONAF, MMA), private actors, and REDD$^+$ mechanisms---with adaptation often framed in market terms. Despite multiple national strategies (DAS 2024 in Germany, ENCCRV in Chile), long-term implementation remains challenged by competing actor coalitions and shifting policy priorities.

However, much of the literature remains siloed either in technical policy analysis or theoretical institutional critique. Comparative, discourse-centered research---focusing on how adaptation is constructed and legitimated in differing contexts---is rare.

\section*{4. Methodology}
\addcontentsline{toc}{section}{4. Methodology}

\subsection*{Research Design}
A qualitative comparative case study, using Germany and Chile as critical cases. Selection is based on their sharply contrasting institutional architectures and parallel exposure to climate-induced forest vulnerabilities.

\subsection*{Data Sources}
Analysis will rely exclusively on secondary sources:

\begin{itemize}
  \item National adaptation strategies (DAS 2024, ENCCRV)
  \item Climate legislation (KSG, Ley Marco de Cambio Climático)
  \item REDD$^+$ documentation and benefit-sharing reports
  \item Reports and statistics from FAO, UNFCCC, World Bank, CONAF, BMEL, Thünen Institute
  \item Academic publications and media documents
\end{itemize}

\subsection*{Analytical Framework}
\begin{enumerate}
  \item \textbf{Discourse Analysis}: Identify dominant narratives, actor coalitions, and framing strategies.
  \item \textbf{Process Tracing}: Unpack institutional change events, focusing on shifts in adaptation policy.
  \item \textbf{Comparative Matrix}: Structure findings across governance paradigm, actors, instruments, and discourse.
\end{enumerate}

\subsection*{Theoretical Orientation}
The thesis synthesizes discursive institutionalism and comparative institutional theory to illuminate how ideational strategies and institutional flexibility drive adaptation in forestry. It offers new insight into whether and how discursive approaches can transfer across divergent institutional landscapes.

\section*{5. Timeline}
\addcontentsline{toc}{section}{5. Timeline}

\begin{longtable}{@{}ll@{}}
\toprule
\textbf{Month} & \textbf{Task} \\ \midrule
October 2025 & Finalize research design and literature review; begin coding scheme \\
November 2025 & Document analysis: Germany (laws, strategies, inventories) \\
December 2025 & Document analysis: Chile (ENCCRV, REDD$^+$, institutional reports) \\
January 2026 & Comparative analysis and synthesis of cases \\
February 2026 & Draft: Introduction, Methods, and Case Study chapters \\
March 2026 & Draft Discussion, Conclusions; editing and revision \\
April 2026 & Final proofreading, submission, and defense preparation \\
\bottomrule
\end{longtable}
