\section*{Introduction}
\addcontentsline{toc}{section}{2. Introduction}

Forests are at the epicenter of the climate crisis, acting as both a crucial 
carbon sink for mitigation and a highly vulnerable ecosystem. The impacts of 
climate change are no longer abstract threats but tangible realities. In Germany, 
four out of five trees were diagnosed as diseased in 2022 due to unprecedented 
drought and insect infestations, turning parts of its famed forests into a 
carbon source (\citealp{BMEL2023}). Similarly, Chile has been ravaged by historic 
"megasequías" and catastrophic wildfires, placing immense pressure on its native 
forests and plantation economy alike (\citealp{FAO2022}). Recent national strategies 
also highlight the urgency of adaptation. The \textit{Plan Nacional de Restauración de 
Paisajes 2021--2030} (\citealp{MMA2021}) sets ambitious targets for restoring degraded 
ecosystems, while the \textit{Ecosystem Restoration Country Dossier} 
(\citealp{MMACONAF2024}) documents ongoing policy innovations. In addition, new 
scientific studies on wildfire risk (\citealp{VidalSilva2025}) and forest hydrology 
(\citealp{Balocchi2023}) provide empirical evidence of the ecological and institutional 
challenges of adaptation in Chile.

Studies of the Central Chile mega-drought (\citealp{Garreaud2020MegaDrought}) and 
analyses of fire regimes (\citealp{McWethy2021FiresChile}) further underscore how 
severe climate stress is feeding into calls for policy innovation in forest governance.

Despite facing analogous biophysical threats, Germany and Chile exhibit vastly 
different governance responses. Germany's approach is characterized by incremental
 adjustments within a highly institutionalized, consensus-oriented system, marked 
 by a decades-long discursive struggle between production and conservation 
 coalitions (\citealp{Winkel2011}). Chile's landscape is defined by the legacy 
 of neoliberalism, resulting in fragmented state authority, powerful private 
 sector actors, and deep-seated conflicts over land and resources, particularly 
 with indigenous Mapuche communities (\citealp{Manuschevich2016}). This divergence 
 presents a critical research puzzle: how do adaptation policies and practices 
 emerge from such dissimilar political and institutional contexts? This question 
 resonates with earlier comparative work identifying considerable divergence in 
 forest governance across regions (\citealp{HowlettRayner2006}) and with recent 
 calls to analyze how institutions addressing forest adaptation evolve within 
 multi-level governance systems (\citealp{KleinschmitChiassonPulzl2023}).

This thesis argues that an answer lies in focusing on the actors who navigate 
these complex environments. It introduces and operationalizes the concept 
of ``ideational bricoleurs'': strategic actors who assemble new governance 
possibilities by creatively combining disparate discourses and available 
institutional resources. By comparing these cases, the research highlights how 
divergent political traditions, actor coalitions, and institutional legacies 
condition not only the dominant policy paradigms but also the very strategies 
available to those seeking to influence them.

\subsection*{Research Aim}

The aim is to identify and compare how ideational bricoleurs in Germany and Chile 
combine discourses and institutional elements to shape forest adaptation governance, 
thereby illuminating the mechanisms of policy change and stasis in two structurally 
distinct contexts.

\subsection*{Research Question}
\begin{quote}
\textit{Which actors function as ideational bricoleurs in forest adaptation governance 
in Germany and Chile, and what strategic combinations of discourses and institutional 
elements do they employ to influence adaptation policies and practices?}
\end{quote}

To answer this, the thesis addresses three sub-questions:
\begin{enumerate}
    \item Who are the key non-state actors (e.g., NGOs, think tanks, community 
    organizations) attempting to shape forest adaptation discourse in each country?
    \item What specific discursive frames (e.g., "climate-smart forestry," 
    "nature-based solutions," "territorial rights") and institutional tools 
    (e.g., market certification, co-management models, new legal articles) do 
    these actors combine in their proposals?
    \item How do the broader institutional contexts of Germany (corporatist, 
    path-dependent) and Chile (neoliberal, fragmented) enable or constrain 
    the strategies and influence of these ideational bricoleurs?
\end{enumerate}

\section*{Literature Review}
\addcontentsline{toc}{section}{3. Literature Review}

\subsection*{Concepts and Theoretical Approaches}

Understanding the politics of climate adaptation requires a framework that connects 
structure and agency, linking the power of ideas to the actions of those who wield them. 
Several traditions provide useful entry points.

First, \textbf{Discursive Institutionalism (DI)} highlights how institutions are not only 
sets of rules but also carriers of ideas and discourses that give them meaning 
(\citealp{Schmidt2008}). DI’s distinction between \textit{coordinative discourse} among 
policy elites and \textit{communicative discourse} for broader publics is essential for 
examining adaptation debates. The related concept of \textbf{Ideational Power} specifies 
the mechanisms of influence, distinguishing between persuasive power through ideas, 
power over ideas, and institutional power in ideas (\citealp{CarstensenSchmidt2016}). 
Recent work has demonstrated how ideational power shapes institutional trajectories 
across governance levels (\citealp{Kleinschmit2024}).

Second, the concept of \textbf{Institutional Bricolage} explains the \textit{how} of 
institutional formation. Originally coined by Lévi-Strauss and later applied to 
natural resource governance (\citealp{Cleaver2001}), bricolage captures the pragmatic 
process whereby actors recombine available rules, norms, and cultural repertoires. 
This has been theorized as a strategy especially visible in contexts of crisis, where 
actors “patch” together elements from different institutional logics to produce novel 
arrangements (\citealp{Carstensen2017}).

However, these two perspectives also operate at different analytical levels. 
While \textbf{Discursive Institutionalism} tends to emphasize the macro-level 
structuring power of ideas and discourses across institutions, 
\textbf{Institutional Bricolage} highlights micro-level agency and the pragmatic 
recombination of locally available norms and rules. This study argues that the 
figure of the \textit{ideational bricoleur} provides a conceptual bridge: it 
captures how actors strategically connect discursive innovation at the macro-level 
with bricolage practices at the local level, thereby making visible how ideas travel, 
hybridize, and reconfigure governance arrangements. Similar efforts to theorize the 
multi-scalar role of ideas in policy change can be found in work on ideational 
institutionalism more broadly (\citealp{BelandCox2011}), which emphasizes how 
different forms of ideational agency operate simultaneously across levels of 
governance.

Third, alternative frameworks provide adjacent perspectives on agency. 
\textbf{Policy entrepreneurs} emphasize the strategic coupling of problems, 
policies, and politics to open “windows of opportunity” 
(\citealp{Kingdon1995,MintromNorman2009}). 
\textbf{Epistemic communities} focus on networks of experts who exert influence 
through shared causal beliefs and professional authority (\citealp{Haas1992}). 
Finally, \textbf{discourse coalition theory} examines how actors coalesce around 
storylines that give meaning to environmental debates (\citealp{Hajer1995,ArtsBuizer2009}). 
These approaches provide valuable insights but often remain either at the abstract level 
of discourse (Hajer, Haas) or at the practical level of institutional strategy 
(entrepreneurship studies), without fully integrating the two. This study positions 
\textbf{ideational bricolage} as a conceptual bridge: it connects discursive 
innovation with institutional patching, thereby capturing how actors creatively 
navigate and reconfigure multilevel governance systems.


\paragraph{Theoretical integration risks.}
Integrating multiple frameworks involves risks of dilution and eclecticism. 
This study mitigates these by clearly delimiting the focus on the 
\textit{ideational bricoleur} as the central synthesis, concentrating on how 
discourse, institutions, and learning coalesce to produce change in forest governance.

\subsection*{Empirical updates in forest adaptation (2020--2024)}

Recent empirical research highlights the urgency and institutional variation of forest adaptation.  
In Germany, the \textit{Report of the Scientific Advisory Board on Forest Policy: 
Adaptation of forests and forestry to climate change} (Thünen Institute, 2022) 
lays out concrete policy instruments and institutional pathways.  
At the European level, the study \textit{Forest-based Climate Change Mitigation 
and Adaptation in Europe} (EFI, Verkerk et al., 2022) reviews recent evidence 
on disturbances, governance instruments, and policy responses.  
Similarly, \textit{Governing Europe’s Forests for Multiple Ecosystem Services: 
Opportunities, Challenges, and Policy Options} (Winkel et al., 2022) emphasizes
 how adaptation, biodiversity, and multifunctionality discourses are increasingly institutionalized.  
The \textit{Forest Europe Implementation Report 2021--2024} documents national 
differences in implementing adaptation measures across signatory countries.  


On the Chilean side, the World Bank’s report \textit{Chile’s Forests: A Pillar 
for Inclusive and Sustainable Growth} (2023) shows how adaptation discourses
are linked to social inclusion and economic policy, illustrating how global 
narratives interact with local institutional legacies. This picture is reinforced 
by national policy initiatives and scientific evidence. The \textit{Plan Nacional de 
Restauración de Paisajes 2021--2030} (\citealp{MMA2021}) and the 
\textit{Ecosystem Restoration Country Dossier} (\citealp{MMACONAF2024}) demonstrate 
institutional pathways for adaptation. Complementing these, recent research on 
wildfire dynamics (\citealp{VidalSilva2025}) and forest hydrology (\citealp{Balocchi2023}) 
highlights the ecological urgency underpinning governance debates.
Analysis of recent fire events and atmospheric conditions (\citealp{McWethy2021FiresChile}) 
and studies of institutional / resilience barriers in Araucanía (\citealp{Banwell2020AraucaniaResilience}) 
show that adaptation is not only a scientific concern but a policy and governance 
challenge with real local constraints.

These contributions complement the conceptual frameworks of Discursive Institutionalism, 
bricolage, policy learning, and resilience thinking by providing concrete evidence for 
comparative analysis.

In addition, recent strands of scholarship have stressed the importance of 
\textbf{policy learning} and \textbf{resilience thinking} in explaining how 
governance systems adapt to climate change. Policy learning approaches highlight 
how actors update their preferences and strategies through processes of social 
interaction and institutional feedback, often across multiple levels of governance 
(\citealp{DuitGalaz2020,JordanHuitema2019}). Similarly, resilience thinking has 
provided a conceptual vocabulary to understand how institutions evolve under 
conditions of uncertainty and disturbance, emphasizing adaptive capacity, 
transformability, and path-dependence. These perspectives complement discursive 
and bricolage approaches by clarifying the cognitive and institutional mechanisms 
through which ideas stabilize or shift. They further reinforce the need for an 
integrative framework---such as the figure of the \textit{ideational bricoleur}---that 
captures how discursive innovation, pragmatic recombination, and adaptive learning 
interact in shaping forest adaptation governance.

\subsection*{Forest Governance and Adaptation Studies}

Scholarship on forest governance has expanded considerably over the past three decades. 
In Europe, analyses have focused on global forest regimes and the rise of 
non-state authority such as certification systems (\citealp{Cashore2004}), 
the institutionalization of discourses in European forest policy 
(\citealp{ArtsBuizer2009}), and the conflicts between multifunctional forestry 
and conservation in Germany (\citealp{Winkel2011}). Comparative studies have 
documented convergence and divergence across governance systems 
(\citealp{HowlettRayner2006}), while multi-level perspectives have shown how 
EU, national, and local arenas interact to shape outcomes (\citealp{Arts2006,Pulzl2013}). 

In Latin America, research has emphasized fragmented authority, contested tenure, 
and the dominance of market-oriented forestry models. Studies have traced how 
such institutions perpetuate inequality and marginalize indigenous knowledge 
(\citealp{LarsonRibot2007,AnderssonPacheco2006,Manuschevich2016}). 
Critical Latin American perspectives underline the embeddedness of governance 
in neoliberal reforms and extractivist paradigms (\citealp{Gudynas2011,Escobar1996}), 
which continue to structure the space for adaptation. These contributions highlight 
that forest governance cannot be disentangled from broader political economies 
and cultural narratives.

Climate change adaptation has entered this field only more recently. In Germany, 
adaptation discourses have largely been absorbed into long-standing forestry versus 
conservation struggles, reinforcing discursive path dependency (\citealp{Winkel2011}). 
In Chile, adaptation has been mediated through plantation-centered institutions 
and narratives privileging growth, often clashing with indigenous and peasant 
worldviews (\citealp{Manuschevich2016}). This reinforces the need for comparative 
research that does not merely juxtapose national cases but interrogates the ways 
in which actors use ideas and institutions to broker change.

\subsection*{Identified Gap and Justification}

While substantial insights exist, comparative research that explicitly theorizes 
the role of actors as brokers of both discourses and institutions remains scarce. 
Existing frameworks—policy entrepreneurship (\citealp{Kingdon1995,MintromNorman2009}), 
epistemic communities (\citealp{Haas1992}), discourse coalitions (\citealp{Hajer1995})—capture 
important aspects of agency, but they either foreground framing and persuasion 
or emphasize strategic action without sufficiently incorporating the enabling 
and constraining role of institutional legacies. 

This thesis introduces the figure of the \textbf{ideational bricoleur} as a 
conceptual synthesis of Discursive Institutionalism and Institutional Bricolage. 
By tracing how actors strategically recombine discourses and institutional tools 
across multiple governance levels, the concept makes visible the mechanisms by 
which ideas travel, hybridize, and occasionally transform entrenched governance 
systems. In doing so, it responds to recent scholarly debates on multilevel forest 
adaptation governance (\citealp{KleinschmitChiassonPulzl2023,Kleinschmit2024}) and 
contributes a comparative North–South perspective (Germany–Chile) that enriches 
theoretical and empirical understandings of adaptation politics.

\section*{Methodology}
\addcontentsline{toc}{section}{4. Methodology}

\subsection*{Research Design}
This thesis applies a qualitative, theory-building comparative case study 
design (\citealp{George2005}). Germany and Chile are selected as ``most different systems'' 
facing similar biophysical pressures but embedded in contrasting political–institutional 
contexts. This design maximizes analytical leverage by exploring how structural 
differences condition actor strategies. 

\textbf{Germany} represents a corporatist welfare state with high institutional density 
and a tradition of consensus-oriented forestry. \textbf{Chile}, in contrast, embodies a 
neoliberal state with fragmented authority, powerful private-sector actors, and persistent 
socio-environmental conflicts. Comparing these two cases enables identification of how 
divergent legacies shape the room for maneuver of ideational actors.

\subsection*{Operationalization of the ``Ideational Bricoleur''}
A core innovation of this study is the operational definition of the 
\textbf{ideational bricoleur}. An actor (organization, coalition, or individual) 
is coded as a bricoleur when all three conditions are observed in the corpus:

\begin{enumerate}
    \item \textbf{Discursive recombination:} The actor explicitly combines elements from 
    at least two distinct or competing discourses (e.g., climate-smart forestry + 
    indigenous territorial rights).
    \item \textbf{Institutional patching:} The actor proposes connecting or adapting 
    institutional tools from different domains (e.g., linking market certification 
    with community-based forest management).
    \item \textbf{Strategic intent:} The actor frames this recombination as a solution 
    to adaptation, signaling intent to influence policy trajectories rather than merely 
    describe problems.
\end{enumerate}

If all three criteria are met in at least two independent documents, the actor will 
be categorized as an \textit{ideational bricoleur}. This threshold guards against 
spurious identification.

\subsection*{Data Collection and Sources}
The analysis will rely on a document corpus produced between 2015 and 2025, 
covering the post-Paris Agreement decade of intense adaptation policy development. 
The corpus includes:

\begin{itemize}
    \item \textbf{Core policy documents:} National adaptation strategies 
    (DE: DAS 2024; CL: ENCCRV, Plan de Adaptación Forestal) and climate legislation 
    (DE: KSG; CL: Ley Marco de Cambio Climático).
    \item \textbf{Grey literature:} Policy briefs, position papers, and manifestos 
    from NGOs, industry associations, and professional organizations.
    \item \textbf{Scientific and technical reports:} Outputs from state-affiliated 
    institutes (e.g., Thünen, EFI, CONAF, INFOR).
    \item \textbf{Public communication:} Media articles, press releases, and 
    parliamentary debates reflecting the communicative discourse.
\end{itemize}

\textit{Limitation:} No interviews are included due to time constraints. However, 
triangulation across policy, grey, and media sources captures both elite 
(coordinative) and public (communicative) discourse, providing a robust basis for 
analysis.


\subsection*{Analytical Framework}
The analysis proceeds in three stages, each with clear coding procedures.

\begin{enumerate}
    \item \textbf{Stage 1: Actor and Frame Identification.}  
    Using frame analysis, all documents will be coded for:
    \begin{itemize}
        \item \textit{Diagnostic frame:} Problem definition (e.g., ``forest death'', 
        ``violation of rights'').
        \item \textit{Prognostic frame:} Proposed solutions (e.g., 
        ``close-to-nature forestry'', ``territorial co-management'').
        \item \textit{Motivational frame:} Calls to action (e.g., science-based 
        legitimacy, justice, efficiency).
    \end{itemize}
    Software: NVivo or MAXQDA will be used for systematic coding.  
    Coding will follow a mixed deductive–inductive approach: deductive categories 
    from the literature (DI, bricolage, policy entrepreneurship) and inductive 
    codes emerging from the texts.

    \item \textbf{Stage 2: Identifying Ideational Bricolage.}  
    Actors will be flagged as bricoleurs if their prognostic frames meet the three 
    operational criteria. Coding memos will record evidence for each case. 

    Example: If an NGO document combines \textit{Nature-Based Solutions} (global discourse) 
    with \textit{national subsidy schemes} (institutional tool) and \textit{customary law} 
    (local institution), it is coded as a bricolage strategy.

    \item \textbf{Stage 3: Comparative Matrix.}  
    Results will be synthesized in a structured comparison:

    \begin{center}
    \begin{tabular}{p{3cm}p{5cm}p{5cm}}
    \toprule
    & \textbf{Germany} & \textbf{Chile} \\
    \midrule
    Key bricoleurs & [List of identified actors] & [List of identified actors] \\
    Hybrid frames & [E.g., climate-smart + multifunctionality] & [E.g., NbS + territorial rights] \\
    Institutional tools & [E.g., EU directives + certification] & [E.g., subsidies + customary law] \\
    Constraints/opportunities & [Consensus corporatism, dense institutions] & [Fragmented authority, high inequality] \\
    \bottomrule
    \end{tabular}
    \end{center}
\end{enumerate}

\subsection*{Reflexivity and Researcher Positionality}
As the analysis relies on textual interpretation, the positionality of the researcher 
must be acknowledged. My academic and professional background in Latin America and 
Europe may sensitize me to certain discourses (e.g., justice, neoliberal critique) 
while potentially downplaying others. This reflexive awareness is incorporated into 
coding memos to reduce interpretive bias.

\section*{Timeline}
\addcontentsline{toc}{section}{5. Timeline}

This research will be conducted under an accelerated timeline, requiring a highly structured 
and intensive work plan from mid-September to the end of November 2025. The schedule is 
divided into three phases: 
(I) Foundation and Data Collection, 
(II) Intensive Analysis, and 
(III) Synthesis and Writing.

\begin{longtable}{@{}lll@{}}
\toprule
\textbf{Weeks} & \textbf{Dates (Approx.)} & \textbf{Detailed Tasks and Key Objectives} \\ \midrule
\endfirsthead
\multicolumn{3}{l}{\ldots continued} \\
\toprule
\textbf{Weeks} & \textbf{Dates (Approx.)} & \textbf{Detailed Tasks and Key Objectives} \\ \midrule
\endhead
\multicolumn{3}{r@{}}{Continued on next page \ldots} \\
\endfoot
\bottomrule
\endlastfoot

\multicolumn{3}{l}{\textbf{Phase I: Foundation and Data Collection (September)}} \\ \addlinespace
Week 3 & Sep 15-21 & \textbf{Finalization \& Archiving:} Finalize proposal with supervisor \\
& & feedback. \textbf{Objective:} Build the complete digital archive: \\
& & systematically download and categorize ALL documents \\
& & (laws, reports, papers) for Germany and Chile. Create a \\
& & document tracker (spreadsheet or Zotero). \\ \addlinespace

Week 4 & Sep 22-28 & \textbf{Develop Analytical Framework:} Develop the coding scheme \\
& & for the frame analysis. \textbf{Objective:} Test and refine the \\
& & coding scheme on a sample set of 4-5 key documents to \\
& & ensure its robustness and applicability. \\ \addlinespace

\multicolumn{3}{l}{\textbf{Phase II: Intensive Analysis (October)}} \\ \addlinespace
Week 5 & Sep 29 - Oct 5 & \textbf{Analysis of German Case:} Conduct the full frame analysis \\
& & of the German document corpus. \textbf{Objective:} Draft an \\
& & analytical memo (5-7 pages) summarizing the identified actors, \\
& & frames, and bricolage strategies in Germany. \\ \addlinespace

Week 6 & Oct 6-12 & \textbf{Analysis of Chilean Case (Part 1):} Begin the frame analysis \\
& & of the Chilean document corpus. \\ \addlinespace

Week 7 & Oct 13-19 & \textbf{Analysis of Chilean Case (Part 2):} Complete the analysis for Chile. \\
& & \textbf{Objective:} Draft an analytical memo for Chile, parallel to the \\
& & German one, to facilitate comparison. \\ \addlinespace

Week 8 & Oct 20-26 & \textbf{Comparative Analysis \& Synthesis:} Populate the comparative matrix. \\
& & \textbf{Objective:} Synthesize the key findings and outline the core \\
& & arguments for the Discussion chapter. This is the intellectual \\
& & turning point of the project. \\ \addlinespace

\multicolumn{3}{l}{\textbf{Phase III: Synthesis and Writing (November)}} \\ \addlinespace
Week 9 & Oct 27 - Nov 2 & \textbf{Drafting - Part 1 (Empirical Chapters):} Draft the full \\
& & versions of the case study chapters (Germany and Chile) and the \\
& & Methodology section. \\ \addlinespace

Week 10 & Nov 3-9 & \textbf{Drafting - Part 2 (Theoretical \& Analytical Chapters):} Draft the \\
& & Introduction, Literature Review, and the crucial Discussion chapter. \\ \addlinespace

Week 11 & Nov 10-16 & \textbf{Drafting - Part 3 (Integration):} Draft the Conclusion. \\
& & \textbf{Objective:} Assemble all chapters into a single, complete first \\
& & draft of the thesis. Conduct a first full read-through for \\
& & coherence and argumentative flow. \\ \addlinespace

Week 12 & Nov 17-23 & \textbf{Intensive Revision:} Incorporate supervisor feedback (if any). \\
& & Conduct a deep revision focusing on clarity, argument structure, \\
& & and evidence. Edit for grammar and style. \\ \addlinespace

Week 13 & Nov 24-30 & \textbf{Finalization \& Submission:} Final proofreading. Format \\
& & bibliography and citations. \textbf{Objective:} Perform final format \\
& & checks and submit the thesis by the end-of-November deadline. \\

\end{longtable}