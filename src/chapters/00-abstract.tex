% --- Resumen / Abstract ---
El presente informe evalúa la pre-factibilidad técnico-económica de implementar Sistemas de Captación y Almacenamiento de Agua de Lluvia (SCALL) para riego en un grupo piloto de 5 a 10 establecimientos educacionales municipales en la comuna de Providencia, Región Metropolitana de Chile. El estudio se enmarca en un contexto de \textit{triple tensión hídrica}: (i) una prolongada sequía y variabilidad climática exacerbada por el cambio global; (ii) la creciente presión de la urbanización; y (iii) un ciclo de alzas tarifarias en los servicios sanitarios.1
El análisis aborda tres brechas críticas identificadas en la literatura: la falta de dimensionamiento localizado, la ausencia de una valoración económica parametrizada con datos reales de tarifas y costos, y la carencia de un modelo de decisión replicable para la priorización de proyectos bajo metas municipales. Como contribución principal, este trabajo propone un modelo de decisión reproducible que integra un enfoque técnico riguroso (oferta-demanda) con una evaluación económico-financiera (Payback, VPN) y una priorización estratégica multicriterio. El modelo está diseñado para alinearse con los lineamientos del Sistema Nacional de Inversiones (SNI) de Chile, facilitando la toma de decisiones informadas y la canalización de la inversión pública.
Los resultados preliminares, basados en un análisis de sensibilidad, indican que la implementación de SCALL puede ser económicamente viable y socialmente beneficiosa en establecimientos seleccionados, con un retorno de la inversión que se ve positivamente afectado por las crecientes tarifas del agua potable. El modelo de priorización produce un \textit{score} transparente que permite a las autoridades municipales identificar las iniciativas de mayor impacto con un riesgo controlado.

% (Opcional) Mostrar "Resumen" en el índice:
% \addcontentsline{toc}{chapter}{Resumen}